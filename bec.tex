\documentclass[a4paper,dvipdfmx]{jsarticle}
\usepackage[dvipdfmx]{graphicx}
\usepackage[top=30truemm,bottom=30truemm,left=25truemm,right=25truemm]{geometry} 
\usepackage{amsmath,amssymb,amsthm}
\usepackage{mathrsfs}
\usepackage{otf}
\usepackage[T1]{fontenc}
\usepackage{amsfonts}
\usepackage{mathtools}
\usepackage{extarrows}
\usepackage{bbm,bm}
\usepackage{url}
\usepackage[dvipdfmx]{hyperref}
\usepackage{physics}
\usepackage{here}
\usepackage{thmtools}
\usepackage{hyperref}
\newcommand{\e}[1]{{\rm{e}}^{#1}}
\allowdisplaybreaks[4]
\begin{document}
\title{理想Bose気体についてのメモ}
\author{Kyon}
\date{\today}
\maketitle

理想Bose気体についてのあれやこれやのまとめ.

\section{おことわり}
めちゃくちゃ雑な計算のメモです.詳しくは教科書を読みましょう.


\section{準備}
この文書では,$\sigma$を
\begin{equation}
\sigma=\begin{cases}
  +1 & \text{for Boson} \\
  -1 & \text{for Fermion} 
\end{cases}
\end{equation}
と定義します\footnote{意味としては置換演算子に対する固有値なのですが,この点は以降はあまり意識しなくても良いです.}.


さて,このように定義するとFermi分布関数,及びBose分布関数は

\begin{equation*}
  f_{\sigma}(\epsilon)=\frac{1}{\e{\beta(\epsilon-\mu)}-\sigma}
\end{equation*}
と書くことができます.Fermion$(\sigma=-1)$の場合ももちろん非常に重要です(3年前期の統計力学の授業で詳しく扱います)が,ここではBoson$(\sigma=+1)$について詳しく見ていくことにしましょう.

\section{状態密度}

今回扱う系の設定をします.体積が$V=L^3$の立方体の箱の中に,質量$m$,スピン$0$の粒子が$N$個入った系について考えます.さらに,周期的境界条件を課します.

この時,系の波動関数は
\[\psi(\bm{r})=\frac{1}{\sqrt{V}}\e{i\bm{k} \cdot \bm{r}} \]となります.ここで,$\bm{k}=\dfrac{2\pi}{L}(n_{x},n_{y},n_{z})$と定義されています($n_{i}$は整数).すると,エネルギー分散関係は$\epsilon(\bm{k})=\dfrac{\hbar^2\bm{k}^2}{2m}$で与えられます.この時,系の状態密度は
\begin{equation}
  D(\epsilon)=\frac{V}{4\pi^{2}}\qty(\frac{2m}{\hbar^2})^{\frac{3}{2}}\epsilon^{\frac{1}{2}}
\end{equation}
となります.スピン$0$であることに注意.さらにこれを体積$V$で割った

\begin{equation}
  \nu(\epsilon)=\frac{1}{4\pi^{2}}\qty(\frac{2m}{\hbar^2})^{\frac{3}{2}}\epsilon^{\frac{1}{2}}
\end{equation}
も用意しておくと楽です.


\section{粒子数の計算,積分近似}



\end{document}